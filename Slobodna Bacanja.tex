% !TEX encoding = UTF-8 Unicode

\documentclass[a4paper, 12pt]{article}

\usepackage[T2A]{fontenc} % enable Cyrillic fonts
\usepackage[utf8x,utf8]{inputenc} % make weird characters work
\usepackage[serbian]{babel}
%\usepackage[english,serbianc]{babel}
\usepackage{amssymb}

\usepackage{color}
\usepackage{url}
\usepackage[unicode]{hyperref}
\hypersetup{colorlinks,citecolor=green,filecolor=green,linkcolor=blue,urlcolor=blue}

\begin{document}

\title{Slobodna bacanja}

\author{Matija Miličević, Jovana TODO}

\maketitle

\section{Zadatak}

Košarkaš prilikom slobodnog bacanja izbacuje loptu sa visine koja odgovara 5/4 sopstvene visine. Ako zanemarimo trenje vazduha, i ako je ugao bacanja $\theta$, da bi lopte prošle centrom obruča potrebna je brzina izbačaja $v_\theta$. Sa istom brzinom $v_\theta$ lopta će ući u koš i za bliske uglove $[\theta_1, \theta_2]$ (a da ne dodirne obruč). Za zadatu visinu košarkaša h odrediti onaj ugao bacanja $\theta$ koji obezbeđuje maksimalnu toleranciju $\theta_2 - \theta_1$. Rezultate tabelirati za omiljeni košarkaški klub ili reprezentaciju.

\section{Pojmovi}



\end{document}