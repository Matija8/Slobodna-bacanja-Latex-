% !TEX encoding = UTF-8 Unicode

\documentclass[a4paper, 12pt]{article}

\usepackage[T2A]{fontenc} % enable Cyrillic fonts
\usepackage[utf8x,utf8]{inputenc} % make weird characters work
\usepackage[serbian]{babel}
%\usepackage[english,serbianc]{babel}
\usepackage{amssymb}
\usepackage{amsmath}

\usepackage{color}
\usepackage{url}
\usepackage[unicode]{hyperref}
\hypersetup{colorlinks,citecolor=green,filecolor=green,linkcolor=blue,urlcolor=blue}

\begin{document}

\title{Slobodna bacanja}

\author{Matija Miličević, Jovana Rađenović}

\maketitle

\section{Zadatak}

Košarkaš prilikom slobodnog bacanja izbacuje loptu sa visine koja odgovara 5/4 sopstvene visine. Ako zanemarimo trenje vazduha, i ako je ugao bacanja $\theta$, da bi lopte prošle centrom obruča potrebna je brzina izbačaja $v_\theta$. Sa istom brzinom $v_\theta$ lopta će ući u koš i za bliske uglove $[\theta_1, \theta_2]$ (a da ne dodirne obruč). Za zadatu visinu košarkaša h odrediti onaj ugao bacanja $\theta$ koji obezbeđuje maksimalnu toleranciju $\theta_2 - \theta_1$. Rezultate tabelirati za omiljeni košarkaški klub ili reprezentaciju.

\section{Pojmovi}

Osnovni pojmovi:\\
$h$ - visina košarkaša koji izvodi slobodna bacanja;\\
$h_k$ - visina koša;\\
$\theta$ - ugao izbačaja;\\
$v_\theta$ - brzina izbačaja za ugao $\theta$ pri kojoj lopta prolazi centrom obruča;\\
$d$ - rastojanje između projekcija košarkaša i centra obruča na parket (rastojanje koša od košarkaša u odnosu na $x$ osu);\\
$R_L$ - poluprečnik lopte;\\
$R_O$ - poluprečnik obruča;\\

*(DIJAGRAM SA ČIČA GLIŠAMA AKO MOGUĆE!!!)\\


\section{Prvi model}

Podrazumevaćemo:
\begin{itemize}

\item da je poluprečnik lopte $R_L$ manji od poluprečnika obruča $R_O$:
\[R_L<R_O\] %(3.1.1)

\item da je $\theta$ između $0^0$ i $90^0$:\\
\[0 < \theta < \pi/2\]

\item da je visina koša $h_k$ veća od visine izbačaja košarkaša $\dfrac{_5}{^4}h$:\\
\[h_k > \dfrac{_5}{^4}h\]  %\hfill \centerline{}
(ovo nije obavezno ali olakšava crtanje slike)\\

\end{itemize}

%*(OVDE IDE DIJAGRAM!!!)\\


Posmatraćemo centar lopte kao materijalnu tačku $(x,y)$ koja počinje svoj put iz tačke $(x_0,y_0) = (0,\dfrac{_5}{^4}h)$. Sile koje deluju na loptu su inicijalna sila šuta pod uglom $\theta$ u odnosu na parket i gravitacija.\\

*(OVDE IDE DIJAGRAM!!!)\\

Da bi lopta prošla centrom obruča poluprečnika $R_O$ mora biti ispunjeno da je $(x,y) = (d,h_k)$ pri padu lopte (lopta je na putanji prošla tačku maksimalne visine $y_{max}$;
Znamo iz postavke zadatka da košarkaš može da dobaci obruč).\\

Pošto su visina koša i košarkaša konstantne veličine možemo da transliramo ceo sistem tako da početna tačka centra lopte bude $(x_0,y_0) = (0,0)$, a tačka centra obruča $C = (d,h_k-\dfrac{_5}{^4}h)$.
Zbog jednostavnosti zapisa uzećemo da je $H = h_k-\dfrac{_5}{^4}h$, odnosno $(x_f,H) = (d,H)$, gde $x_f$ predstavlja $x$ koordinatu lopte u trenutku kada lopta pada do visine obruča.\\


*(OVDE IDE POPRAVLJENI DIJAGRAM!!! +naznačiti $y_{max}$)\\

Po ovoj slici možemo primetiti da je polazni problem u stvari jedna vrsta problema kosog hica.

\subsection{Kretnja lopte}

*(TROUGAO SA SILAMA?)\\

%Možda ovde ubaciti neke izvode tipa y_prim, y_prim_prim

Po slici možemo da vidimo da na $x$ koordinatu ne utiče nijedna sila (otpor vazduha zanemarujemo), a pošto se lopta kreće brzinom zadatom pri šutu dolazimo da zaključka da važi:\\

\[x(t) = x_0 + v_{\theta x}*t\]

Gde su: $x_0 = 0$; $v_{\theta x} = v_\theta * \cos \theta$; odnosno:

\[x(t) = v_\theta * \cos \theta*t\]

Sa druge strane na $y$ koordinatu sve vreme utiče gravitacija. Uz početnu brzinu zadatu šutom možemo da zaključimo da važi:

\[y(t) = y_0 + v_{\theta y}*t - \dfrac{g*t^2}{2}\]

gde je $g \approx 9.81\dfrac{_m}{^{s^2}}$. Pošto važi da su: $y_0 = 0$; $v_{\theta y} = v_\theta * \sin \theta$; sledi:

\[y(t) = v_{\theta}* \sin \theta*t - \dfrac{g}{2}*t^2\]

\subsection{Uslovi za prolazak lopte kroz obruč}

*(SLIKA:Lopta kako ulazi u obruć)\\


\end{document}